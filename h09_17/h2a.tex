
\documentclass{article}

\usepackage{listings}
\usepackage{graphicx}
\usepackage{hyperref}

\hypersetup{colorlinks,linkcolor=,urlcolor=blue}

\begin{document}

\begin{verbatim}
Our goal is to maximize profit of the hotel.
Assume profit to be room_price*occupancy (not counting expenses).
Then make 18 (20-2) plots, where x is corresponding factor and y is profit.
Some relations were found:
    (8) more businessmen share ~ a bit more profit;
    (9) more tourists share ~ a bit less profit;
    (18) more length of stay ~ more profit;
    (20) conventions ~ a bit more profit;
It is very easy to misinterpret, what are reasons and what are consequences here.
Suggestions to improve profit can be for example:
- to organize conventions for businessmen.
- to give discounts for long-stays.
Seasons does not seem to affect situation, 
while some annual events can exist but are not discoverable on one-year range.
Price vs occupancy plot does not show any relation,
while occupancy is changing in a wider range than price.
May be the price is already highly tuned, or it is random.
Before changing prices we'd better look at the competitors' strategy.
It seems that the hotel is located in business district,
and profit-improvement-strategies need to be performed taking into consideration 
other information like flights annulations or bank holidays.
\end{verbatim}

\begin{figure}[h!]
\centering
\includegraphics[width=80mm]{out/f18_pro.png}
\caption{length of stay to profit}
\end{figure}

\clearpage

\lstinputlisting[language=Python,numbers=left]{h2a.py}

\url{https://github.com/rewlad/ttu_andm}

\end{document}